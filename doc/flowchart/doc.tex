\documentclass[parskip=half]{scrartcl}

\usepackage{polyglossia}
\setmainlanguage{german}

\usepackage{fontspec}
\setmainfont{Linux Libertine O}

\usepackage{tikz}
\usetikzlibrary{positioning}

\tikzset{
    state/.style={draw, rectangle, thick, rounded corners, align=center, text width=4cm}
}

\begin{document}
\section{Collection of ideas}
  \subsection{Automatic error correction}
  I have some mistakes i do alot, like forgetting to load a package or a tikz library or something. This is nearly all the time easy to fix and, more
  important, easy to detect automatically. I see two ways for automating this task, both should be used together: 1. check the difference in edits BEFORE
  compiling and add packages and libraries for inserted code. 2. parse error messages, insert missing code and recompile without showing the error to the
  user. There should be a message like "added package 'foo' because of line 13: '...'".
\section{Program flowchart}
  \begin{center}
    \begin{tikzpicture}
      \node[state] (init) { Initialize program };
      \node[state, below=of init] (loop) { Loop };
    \end{tikzpicture}
  \end{center}
\end{document}

% vim: ft=tex tw=0
